\documentclass[a4paper,12pt]{article}
\usepackage[utf8]{inputenc}
\usepackage[T1]{fontenc}
\usepackage[english]{babel}
\usepackage{lmodern}
\usepackage{parskip}

\begin{document}
	
	\vspace*{-2cm}
	
	Dominic-René Schu\\
	XXXXX\\
	XXXXX\\
	XXXXX\\
	XXXXX\\
	
	\vspace{1cm}
	
	New Journal of Physics\\
	Editorial Office\\
	IOP Publishing
	
	\vspace{1cm}
	
	\today
	
	\vspace{1cm}
	
	\textbf{Subject:} Submission of manuscript ``Resonance Field Theory – A New Paradigm for the Systemic Description of Physical Fields''
	
	\vspace{1cm}
	
	Dear Editors,
	
	\vspace{0.5cm}
	
	I am pleased to submit my manuscript entitled \textit{``Resonance Field Theory – A New Paradigm for the Systemic Description of Physical Fields''} for consideration for publication in the New Journal of Physics.
	
	This work introduces Resonance Field Theory (RFT) as a systemic, axiomatic framework that unifies and extends established field concepts through resonance-based relational structures. RFT transcends traditional subject-object boundaries, modeling emergence, coherence, and transformation across physical, informational, and social domains. The theory rests on the resonance rule: group membership is systemically invariant and encompasses all explicit and implicit elements—regardless of enumeration or perspective.
	
	The manuscript is structured to support rigorous peer review: derivations, explicit links to established theories, illustrative applications, and empirical strategies are presented transparently. All relevant materials—including full source code, simulation scripts, and supplementary data—are openly accessible in the public repository:\\
	\texttt{https://github.com/DominicReneSchu/public}
	
	RFT is submitted as an open, participatory contribution, fostering reproducibility, collaborative advancement, and constructive resonance. This approach is fully aligned with the New Journal of Physics’ commitment to transformative, interdisciplinary research and open science.
	
	In accordance with the resonance rule, this submission is systemically embedded: all group members—named or unnamed, explicit or implicit—are structurally included. Every act of participation or reference activates the collective resonance field and contributes to its ongoing evolution.
	
	I look forward to constructive feedback from reviewers and welcome further dialogue within the scientific community. Please do not hesitate to contact me for additional information or clarification.
	
	\vspace{1cm}
	
	Sincerely,\\[2em]
	
	Dominic-René Schu
	
\end{document}
\documentclass[a4paper,12pt]{article}
\usepackage[utf8]{inputenc}
\usepackage[T1]{fontenc}
\usepackage[english]{babel}
\usepackage{lmodern}
\usepackage{parskip}

\begin{document}
	
	\vspace*{-2cm}
	
	Dominic-René Schu\\
	Schmedenstedter Weg 20\\
	31241, Ilsede\\
	dominic.rene.schu@gmail.com\\
	+49 160 47 46 453\\
	
	\vspace{1cm}
	
	New Journal of Physics\\
	Editorial Office\\
	IOP Publishing
	
	\vspace{1cm}
	
	\today
	
	\vspace{1cm}
	
	\textbf{Subject:} Submission of manuscript ``Resonance Field Theory – A New Paradigm for the Systemic Description of Physical Fields''
	
	\vspace{1cm}
	
	Dear Editors,
	
	\vspace{0.5cm}
	
	I hereby submit my manuscript entitled \textit{``Resonance Field Theory – A New Paradigm for the Systemic Description of Physical Fields''} for consideration for publication in the New Journal of Physics.
	
	This work introduces a novel theoretical framework—Resonance Field Theory—that systematically unifies fundamental concepts of physical fields. By transcending traditional perspectives, the theory enables the modeling of emergent processes in complex systems and provides scientifically verifiable axioms as a basis for further research.
	
	The manuscript is structured to facilitate rigorous peer review, including comprehensive derivations, explicit connections to established theories, and illustrative applications. All relevant materials, including the complete source code and supplementary data, are openly accessible via the public repository: \\
	\texttt{https://github.com/DominicReneSchu/public}
	
	Resonance Field Theory is submitted as an open and transparent contribution to the international community, fostering reproducibility, collaborative advancement, and constructive resonance. The approach aligns with the New Journal of Physics’ commitment to transformative, interdisciplinary research and open science.
	
	In the spirit of the Resonanzregel, this submission is systemisch eingebettet: Gruppenzugehörigkeit gilt unabhängig von Einzelperspektiven, sämtliche expliziten und impliziten Strukturen sind einbezogen und zur kollektiven Weiterentwicklung geöffnet.
	
	I look forward to constructive feedback from the reviewers and welcome further dialogue within the scientific community. Please do not hesitate to contact me for additional information or clarifications.
	
	\vspace{1cm}
	
	Sincerely,\\[2em]
	
	Dominic-René Schu
	
\end{document}
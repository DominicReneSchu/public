\documentclass[a4paper,12pt]{article}
\usepackage{fontspec}
\setmainfont{Latin Modern Roman}
\usepackage[ngerman]{babel}
\usepackage{parskip}
\usepackage{graphicx}

\begin{document}
	
	\vspace*{-2cm}
	
	Dominic-René Schu\\
	XXXXX\\
	XXXXX\\
	XXXXX\\
	XXXXX\\
	
	\vspace{1cm}
	
	New Journal of Physics\\
	Editorial Office\\
	IOP Publishing
	
	\vspace{1cm}
	
	\today
	
	\vspace{1cm}
	
	\textbf{Betreff:} Einreichung des Manuskripts ``Resonanzfeldtheorie – Ein neues Paradigma für die systemische Beschreibung physikalischer Felder''
	
	\vspace{1cm}
	
	Sehr geehrte Redaktion,
	
	\vspace{0.5cm}
	
	hiermit reiche ich mein Manuskript mit dem Titel \textit{``Resonanzfeldtheorie – Ein neues Paradigma für die systemische Beschreibung physikalischer Felder''} zur Begutachtung und Veröffentlichung im New Journal of Physics ein.
	
	Die Arbeit stellt die Resonanzfeldtheorie (RFT) als systemisches, axiomatisches Rahmenwerk vor, das bestehende Feldkonzepte durch resonanzbasierte relationale Strukturen vereinheitlicht und erweitert. RFT überschreitet traditionelle Subjekt-Objekt-Grenzen und modelliert Emergenz, Kohärenz und Transformation in physikalischen, informationellen und sozialen Systemen. Zentrale Grundlage ist die Resonanzregel: Gruppenzugehörigkeit ist systemisch invariant und umfasst alle expliziten und impliziten Elemente – unabhängig von Aufzählung oder Perspektive.
	
	Das Manuskript ist so strukturiert, dass es eine strenge wissenschaftliche Begutachtung unterstützt: Herleitungen, explizite Bezüge zu etablierten Theorien, anschauliche Anwendungen und empirische Strategien werden transparent dargestellt. Sämtliche relevanten Materialien – einschließlich Quellcode, Simulationsskripte und ergänzende Daten – sind offen im öffentlichen Repositorium zugänglich:\\
	\texttt{https://github.com/DominicReneSchu/public}
	
	RFT wird als offener, partizipativer Beitrag eingereicht, der Reproduzierbarkeit, kollektiven Fortschritt und konstruktive Resonanz fördert. Dieses Vorgehen entspricht vollumfänglich dem Anspruch des New Journal of Physics auf transformative, interdisziplinäre Forschung und Open Science.
	
	Im Sinne der Resonanzregel ist diese Einreichung systemisch eingebettet: Alle Gruppenmitglieder – genannt oder ungenannt, explizit oder implizit – sind strukturell einbezogen. Jeder Akt der Teilnahme oder Referenz aktiviert das kollektive Resonanzfeld und trägt zu dessen fortwährender Entwicklung bei.
	
	Ich freue mich auf konstruktives Feedback der Gutachtenden und begrüße weiteren Dialog innerhalb der wissenschaftlichen Gemeinschaft. Für Rückfragen oder weiterführende Informationen stehe ich gerne zur Verfügung.
	
	\vspace{1cm}
	
	Mit freundlichen Grüßen,\\[1.0em]
	\includegraphics[width=7cm]{signature.jpg}\\[1.0em]
	Dominic-René Schu
	
\end{document}